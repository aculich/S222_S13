% last edited 4/30/2013 3:55 by PBS
% -*-Mode: LaTeX;-*-

%\documentclass[fleqn,handout]{beamer}
\documentclass[fleqn]{beamer}
\mode<presentation>
{
  \usetheme{CambridgeUS} % clean!
  \usecolortheme{orchid}
\setbeamercolor{block title}{use=structure,fg=black,bg=red!60!black}
\setbeamercolor{block body}{use=structure,fg=white,bg=white!20!black}
%\usetheme{Goettingen}
%  \usetheme{Pittsburgh} % simple, but right flush titles
%  \usetheme{Copenhagen}
%  \usetheme{PaloAlto}
%  \usetheme{Rochester} % very simple, no headings
%  \usetheme{Berkeley} % ugly square items
  \setbeamercovered{transparent}  % or whatever (possibly just delete it)
}
\useinnertheme{rectangles}
\setbeamertemplate{caption}[numbered]
\setbeamerfont{caption}{size=\tiny}
\usepackage{graphicx}
% If you wish to uncover everything in a step-wise fashion, uncomment
% the following command:

%\beamerdefaultoverlayspecification{<+->}
\usepackage{amsmath}
\usepackage{comment}
%\usepackage{caption}
\usepackage{multirow}
\usepackage{graphicx}
\usepackage{url}

\parskip 0.66ex


\begin{document}

\title[] % (optional, use only with long paper titles)
{Predicting YouTube Comedy Slam Winners}

%\author[Author, Another] % (optional, use only with lots of authors)
\author{STAT 222 Class 2013}
\institute{{\tiny STAT \\UC Berkeley}}

% - Use the \inst{?} command only if the authors have different
%   affiliation.
%\author[Author, Another] % (optional, use only with lots of authors)
% - Use the \inst{?} command only if the authors have different
%   affiliation.

% \date % [Short Occasion] % (optional)
\date{7 May 2013 }
\subject{Statistical Machine Learning}


\begin{frame}
\titlepage
\end{frame}

%\begin{frame}
%\frametitle{Outline}
%{\small
%\tableofcontents
%}
% %  % You might wish to add the option [pausesections]
%\end{frame}

%----------------------------------------------------------------------------------
\begin{frame}
\frametitle{YouTube Comedy Slam Data}

	\begin{itemize}
	   \item from UCI ML Repository
	   \item a trial is an ordered pair of video IDs
	   \item data: ordered pair $+$``left'' or ``right'' found funnier by viewer
	   \item order of the pair in each trial was random
	   \item repository has training data and test data
	\end{itemize}

	\begin{beamerboxesrounded}{Goal}
	    Predict which video in a pair will be funnier, from metadata
	\end{beamerboxesrounded}
\end{frame}

%----------------------------------------------------------------------------------
\begin{frame}
\frametitle{Tools}
   \begin{itemize}
        \item Github
        \item IPython notebook
        \item numpy, scipy, matplotlib, nltk, scikit-learn
   \end{itemize}

\end{frame}

%----------------------------------------------------------------------------------
\begin{frame}
\frametitle{Descriptive statistics of training data}
      \begin{itemize}
         \item 912969 records
         \item 18474 distinct videos
         \item ??? distinct video pairs
         \item 359874 distinct ordered pairs of videos
         \item right video won 51.77 \% of the time. $P$-value: ???
         \item viewers often discordant: EXAMPLE
         \item accuracy of Bayes classifier for training data: 73.449\%
         \item for individual videos, directed graph of ``funnier than''
         \item can summarize that graph by PageRank
     \end{itemize}

\end{frame}

%----------------------------------------------------------------------------------
\begin{frame}
\frametitle{PageRank}
      \begin{itemize}
         \item Assign each video (node) in the graph a numerical value
         between $0$ to $1$, known as its \textsl{PageRank}.
         \begin{center}
         \begin{tabular}{| l |}
         \hline
         At $t=0$, $PR(p_{i};0) = \frac{1}{N}$, $N$ is the total number of nodes. \\ \hline
         $PR(p_{i};t+1) = \frac{1-d}{N} + d \sum_{p_{j} \in M(p_{i})} \frac{PR(p_{j};t)}{L(p_{j})}$ \\ \hline
         Algorithm ends when $|PR(t+1) - PR(t)| \le \epsilon$ \\ \hline
         $d$ here is a damping factor, default value is $0.85$ in scikit-learn. \\
         \hline
         \end{tabular}
         \end{center}
     \end{itemize}

\end{frame}

%----------------------------------------------------------------------------------
\begin{frame}
\frametitle{Acquiring metadata}
   \begin{itemize}
       \item queried YouTube with video IDs using Google APIs w/i Python
       \item speedbumps: had to throttle the requests
       \item stored ``snapshot'' data as pickled Python dict
   \end{itemize}
\end{frame}

%----------------------------------------------------------------------------------
\begin{frame}
\frametitle{Prediction using metadata: Feature selection}
      \begin{itemize}
          \item used mutual information to screen potential features
          \item quantitative metadata (e.g., \#views, rating, \#raters) unhelpful
          \item comments more promising
      \end{itemize}

\end{frame}

%----------------------------------------------------------------------------------
\begin{frame}
\frametitle{Issues with comments}
    \begin{itemize}
        \item inconsistent spellings (hillarous, \textellipsis)\\
                Tried Norvig's spelling corrector, ported to AWS. \\
                Not accurate on this corpus\\
                  (examples: ``youtube'' $\rightarrow$ ``couture'' ADD MORE)

        \item used nltk Porter stemmer, but not very accurate\\
                   Examples: (TO DO)

        \item text-speak:  LOL, LOLOLOL, HAHA, HAHAHAH, FOFL, ROFL, LMAO\\
                 Used RegExp to standardize these

        \item emoticons: used RegExps to replace happy faces (e.g., ``:-]'') with ``happyFace'' 
                  before stripping other punctuation

    \end{itemize}
\end{frame}

%----------------------------------------------------------------------------------
\begin{frame}
\frametitle{Bags of Bags of Words}
     \begin{itemize}
         \item ordered pair of videos reduced to two ``bags of bags of words''  \\
                  each comment is a bag, each video has a bag of bags

         \item features derived from bags of bags:
             \begin{itemize}
                 \item presence of a word in any comment
                 \item frequency of comments with a given word
                 \item relative frequency of comments with a given word
                 \item  logOdds of frequencies and relative frequencies
                 \item etc.
              \end{itemize}
         
         \item another derived feature: PageRank predicted by linear regression\\ 
                  predictor used (among other things) $\arctan$(difference in LOL counts)
     
     \end{itemize}
\end{frame}

%----------------------------------------------------------------------------------
\begin{frame}
\frametitle{Classifiers}

     \begin{beamerboxesrounded}{Logistic regression}
         Explain
     \end{beamerboxesrounded}
     
     \begin{beamerboxesrounded}{CART}
         Explain
      \end{beamerboxesrounded}

\end{frame}

%----------------------------------------------------------------------------------
\begin{frame}
\frametitle{Performance on the training set}

\end{frame}

%----------------------------------------------------------------------------------
\begin{frame}
\frametitle{The Test data}
    \begin{itemize}
        \item Accuracy of the Bayes classifier: 0.6946
        \item Left/right bias in the test set: ???
    \end{itemize}

\end{frame}

%----------------------------------------------------------------------------------
\begin{frame}
\frametitle{Performance on test data}
    \begin{itemize}
        \item Bayes classifier from training data applied to test data
        \item logistic regression
        \item CART
    \end{itemize}

\end{frame}

%----------------------------------------------------------------------------------
\begin{frame}
\frametitle{Conclusions}
    the problem and the data

\end{frame}

%----------------------------------------------------------------------------------
\begin{frame}
\frametitle{Lessons learned}
   Tools \& environment (github, IPython, etc.)


\end{frame}
\end{document}